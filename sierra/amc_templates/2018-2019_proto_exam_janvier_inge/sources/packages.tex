\usepackage[utf8x]{inputenc}    
\usepackage[T1]{fontenc}

\usepackage{ifthen}

\ifthenelse{\equal{\OPTcatalogue}{true}}{
\usepackage[francais,completemulti,box,insidebox,indivanswers]{automultiplechoice}
}{
\usepackage[francais,completemulti,box,insidebox]{automultiplechoice}
}


\usepackage{pdfpages}

\usepackage{fancyhdr} % Pour pouvoir changer le style des numéros de page
\pagestyle{fancy}
\fancyhf{}

\usepackage{wrapfig} % Pour permettre la création de figure "enveloppées" par le texte

\usepackage{graphicx}%pour utiliser includegraphics
\usepackage[shortlabels]{enumitem}%pour changer le style des enumerate

\usepackage{tikz}
\usepackage{circuitikz}%pour dessiner
\usetikzlibrary{babel}
\usetikzlibrary{arrows}
\usepackage{pgfplots}%pour pouvoir aussi utiliser l'environnement axis
\usepackage{bm}%pour pouvoir mettre les symboles mathématiques en gras

\usepackage{pgffor}

\usepackage{epstopdf}

\usepackage{qrcode} % Pour générer des QR-codes (compatible avec PDFLaTeX)

%% Réduire les marges
\geometry{hmargin=2cm,headheight=2cm,headsep=.3cm,footskip=1cm,top=2.6cm,bottom=2.5cm}   

\usepackage{multicol}
\usepackage{changepage} % pour ajuster les marges localement dans le document
  
\usepackage{fp} % pour faire des maths dans LaTeX
\usepackage{amsmath} % Extension symboles mathématiques (pour les phaseurs dans le cas présent)
\usepackage{bm} % "bold mathematics" pour mettre les symboles mathématiques en gras (lettres grecques en particulier). Commande \bm{}.

\usepackage{keyval}% http://ctan.org/pkg/keyval
\usepackage[binary,squaren,Gray,cdot]{SIunits}%Pour écrire les volts et les décibels


\usepackage{forloop} % pour faire des boucles "for" en latex


% pour placer le numéro de l'interro
%\usepackage[pscoord]{eso-pic}% The zero point of the coordinate system is the lower left corner of the page (the default).
\newcommand{\placetextbox}[3]{% \placetextbox{<horizontal pos>}{<vertical pos>}{<stuff>}
  \setbox0=\hbox{#3}% Put <stuff> in a box
  \AddToShipoutPictureFG*{% Add <stuff> to current page foreground
    \put(\LenToUnit{#1\paperwidth},\LenToUnit{#2\paperheight}){\vtop{{\null}\makebox[0pt][c]{#3}}}%
  }%
}%


%%%% Définitions pour la section
\def\inge{PHYS-S-1001 INGE}
\def\irci{PHYS-H-100 IRCI}
\def\irar{PHYS-H-100 IRAR}
\def\irbi{PHYS-H-101 IRBI}

%%%%% Définitions pour les chapitres
\def\pathtoquestions{../../Questions/}
\def\TH{\pathtoquestions questionsTH.tex}
\def\ES{\pathtoquestions questionsES.tex}
\def\MS{\pathtoquestions questionsMS.tex}
\def\EM{\pathtoquestions questionsEM.tex}
\def\OO{\pathtoquestions questionsOO.tex}
\def\PM{\pathtoquestions questionsPM.tex}
\def\LA{\pathtoquestions questionsL1.tex}
\def\LB{\pathtoquestions questionsL2.tex}
\def\LC{\pathtoquestions questionsL3.tex}
\def\LD{\pathtoquestions questionsL4.tex}
\def\LE{\pathtoquestions questionsL5.tex}
\def\LF{\pathtoquestions questionsL6.tex}
\def\test{questionsTEST.tex}

% Définir les chemins d'accès aux figures
\graphicspath{{../fig/corefig/}{../fig/}}

% Choisir une seed pour le mélange aléatoire des questions
\AMCrandomseed{152738} %1527384