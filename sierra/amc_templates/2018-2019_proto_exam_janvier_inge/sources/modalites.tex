% Modalités
\noindent\textbf{Modalités :}\\
\begin{itemize}
	\item Vérifiez que cette brochure comporte \nbpages ~pages.
	\item Indiquez immédiatement votre matricule, votre nom et votre prénom ci-dessus.
	\item Maintenez agrafées les feuilles du présent questionnaire et du feuillet de brouillon.
	\item N’utilisez que les feuilles de brouillon que l’on met à votre disposition.
	\item Tout résultat indiqué en dehors des cases de réponse ne sera pas considéré.
	\item N'écrivez pas les développements qui vous ont menés à votre
	réponse. Seules les réponses demandées dans les énoncés des sous-questions peuvent apparaître dans les cadres prévus à cet effet.
	\item Les réponses analytiques doivent être exprimées en utilisant les symboles des variables
	notés en gras dans les énoncés correspondant. Pour les variables qui ne se trouvent pas dans l'énoncé, le choix du symbole est libre.
	\item Sauf mention contraire, les valeurs numériques doivent être exprimées avec 3 chiffres significatifs et dans les unités du SI.
	\item Vous pouvez écrire au crayon sauf pour les réponses nécessitant un encodage.
	\item \textbf{Les téléphones portables et les appareils connectés sont interdits pendant l'examen et doivent être laissés dans vos sacs éteints en bas de l'auditoire.}
\end{itemize}

	\vspace{1cm}
\noindent\textbf{Remarques :}
\begin{itemize}
	\item Sauf mention contraire l’accélération de la gravitation terrestre est supposée être égale à $9,81 \, \meter / \second^2$.
	\item On négligera toute source de frottement et de dissipation de l’énergie.
	\item Les ressorts sont supposés de masse négligeable.
	\item On considérera les masses atomiques relatives suivantes : H : 1, C : 12, N : 14, O : 16, Fe : 56
\end{itemize}

\vspace{1cm}
\begin{center}
	\textbf{La durée du test est de \duree.}
\end{center}